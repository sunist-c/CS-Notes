% Attention: Font "FiraCode N(erd) F(ont)" have different name between macOS and windows

%!TEX encoding = UTF-8 Unicode
\documentclass[a4paper,UTF8,heading=false,12pt]{article}
\usepackage[heading = false]{ctex}
\usepackage{graphicx}

% package
\usepackage{xeCJK}
\usepackage{blindtext}
\usepackage{palatino}
% \usepackage{titlesec}
\usepackage{needspace}
\usepackage[toc,page]{appendix}
\usepackage{metainfo}
\usepackage[pagestyles,raggedright]{titlesec}
\usepackage[xetex,colorlinks=false,hidelinks,pdfstartview=FitV]{hyperref}
\usepackage{amsmath,amssymb}
\usepackage{newpxtext,newpxmath}
\usepackage{listings}
\usepackage{fontspec}
\usepackage{xcolor-material}
\usepackage{enumitem}
\usepackage{booktabs}
% configuration
\fontsize{16pt}{\baselineskip}
%\setmonofont{FiraCode NF}
\CJKfamily{zhsong}
%\addtolength{\parindent}{3cm}
% code configuration
\definecolor{codegreen}{rgb}{0,0.6,0}
\definecolor{codegray}{rgb}{0.5,0.5,0.5}
\definecolor{codepurple}{rgb}{0.58,0,0.82}
\definecolor{backcolour}{rgb}{0.95,0.95,0.92}
\lstdefinestyle{customc}{
    belowcaptionskip=1\baselineskip,
    breaklines=true,
    frame=none,
    xleftmargin=2em,
    language=C++,
    showstringspaces=false,
    basicstyle=\footnotesize\ttfamily,
    keywordstyle=\bfseries\color{green!40!black},
    commentstyle=\itshape\color{purple!40!black},
    numberstyle=\fontspec{FiraCode NF}\color{darkgray},
    numbers=left,
    identifierstyle=\color{blue},
    stringstyle=\color{orange},
}

\usepackage[paperwidth=210mm, paperheight=297mm, margin=20 mm]{geometry}

\newcommand\subtitle[1]{{\small #1}} 

\begin{document}
\title{
    操作系统原理实验报告 \\
    \subtitle{实验三: 页面置换算法模拟}
}
\author{陈羿羽 \thanks{陈羿羽:西南大学2019级计算机科学与技术4班 - 222019603193014}}
\maketitle

\begin{abstract}
    本次实验的目标为:

    设计一个虚拟存储区和内存工作区,编程序演示下属算法的具体实现过程,并计算访问命中率:

    要求设计主界面以灵活选择某算法,且以下算法都要实现。

    \begin{enumerate}
        \item 最佳置换算法(OPT):将以后永不使用的或许是在最长(未来)时间内不再被访问的页面换出。
        \item 先进先出算法(FIFO):淘汰最先进入内存的页面,即选择在内存中驻留时间最久的页面予以淘汰。
        \item 最近最久未使用算法(LRU):淘汰最近最久未被使用的页面。
    \end{enumerate}
\end{abstract}

\newpage

\section{实验要求}

\begin{enumerate}
    \item 用C语言编写OPT、FIFO、LRU置换算法。
    \item 熟悉内存分页管理策略。
    \item 了解页面置换的算法。
    \item 掌握一般常用的调度算法。
    \item 根据方案使算法得以模拟实现。
    \item 锻炼知识的运用能力和实践能力。
\end{enumerate}

\section{实验报告}

\subsection{算法源代码}

\subsection{主要流程图}

\subsection{运行结果截图}

\subsection{实验总结}

\section{基本思想}

选择置换算法,先输入所有页面号,为系统分配物理块,依次进行置换:

\subsection{OPT基本思想}

是用一维数组page[pSIZE]存储页面号序列,memery[mSIZE]是存储装入物理块中的页面。数组next[mSIZE]记录物理块中对应页面的最后访问时间。每当发生缺页时,就从物理块中找出最后访问时间最大的页面,调出该页,换入所缺的页面。

\subsection{FIFO基本思想}

是用队列存储内存中的页面,队列的特点是先进先出,与该算法是一致的,所以每当发生缺页时,就从队头删除一页,而从队尾加入缺页。或者借助辅助数组time[mSIZE]记录物理块中对应页面的进入时间,每次需要置换时换出进入时间最小的页面。

\subsection{LRU基本思想}

是用一维数组page[pSIZE]存储页面号序列,memery[mSIZE]是存储装入物理块中的页面。数组flag[10]标记页面的访问时间。每当使用页面时,刷新访问时间。发生缺页时,就从物理块中页面标记最小的一页,调出该页,换入所缺的页面。

\end{document}