%!TEX encoding = UTF-8 Unicode
\documentclass[a4paper,UTF8,heading=false,12pt]{article}
\usepackage[heading = false]{ctex}
\usepackage{graphicx}

% package
\usepackage{xeCJK}
\usepackage{blindtext}
\usepackage{palatino}
% \usepackage{titlesec}
\usepackage{needspace}
\usepackage[toc,page]{appendix}
\usepackage{metainfo}
\usepackage[pagestyles,raggedright]{titlesec}
\usepackage[xetex,colorlinks=false,hidelinks,pdfstartview=FitV]{hyperref}
\usepackage{amsmath,amssymb}
\usepackage{newpxtext,newpxmath}
\usepackage{listings}
\usepackage{fontspec}
\usepackage{xcolor-material}
\usepackage{enumitem}
\usepackage{booktabs}
% configuration
\fontsize{16pt}{\baselineskip}
\setmonofont{FiraCode Nerd Font}
\CJKfamily{zhsong}
%\addtolength{\parindent}{3cm}
% code configuration
\definecolor{codegreen}{rgb}{0,0.6,0}
\definecolor{codegray}{rgb}{0.5,0.5,0.5}
\definecolor{codepurple}{rgb}{0.58,0,0.82}
\definecolor{backcolour}{rgb}{0.95,0.95,0.92}
\lstdefinestyle{customc}{
    belowcaptionskip=1\baselineskip,
    breaklines=true,
    frame=none,
    xleftmargin=2em,
    language=C++,
    showstringspaces=false,
    basicstyle=\footnotesize\ttfamily,
    keywordstyle=\bfseries\color{green!40!black},
    commentstyle=\itshape\color{purple!40!black},
    numberstyle=\fontspec{FiraCode Nerd Font}\color{darkgray},
    numbers=left,
    identifierstyle=\color{blue},
    stringstyle=\color{orange},
}

\usepackage[paperwidth=210mm, paperheight=297mm, margin=20 mm]{geometry}

\newcommand\subtitle[1]{{\small #1}} 

\begin{document}
    \title{
        数据库系统概论实验报告 \\
        \subtitle{实验四: 两种方式向数据表添加、导入和导出数据}
    }
    \author{陈羿羽 \thanks{陈羿羽:西南大学2019级计算机科学与技术4班 - 222019603193014}}
    \maketitle

    \begin{abstract}
        本次实验的目标为:
        \begin{enumerate}
            \item 使用数据库管理系统SQL Server 或者MySQL操作数据库。
            \item 掌握创建、查看、修改和删除数据表的基本操作。
        \end{enumerate}
    \end{abstract}

    \newpage

    \section{实验准备}

    \begin{enumerate}
        \item 一台PC机或笔记本电脑
        \item 能够上网
    \end{enumerate}

    \section{实验要求}

    \begin{enumerate}
        \item 查资料观看数据库管理系SQL Server 或者MySQL的数据表数据添加、导入和导出视频。
        \item 使用视图方式和命令方式向数据表添加、导入和导出数据。
        \item 学习了解数据表字段的基本类型。
    \end{enumerate}

    \section{实验内容}

    \subsection{视图方式建立数据表}

    向实验三创建的数据表添加数据:
    \begin{enumerate}
        \item 把学生成绩表和班级信息表的数据,自己各构造10条存在名字为“学生成绩.txt”和“班级信息.txt”的文本文件中,然后导入到数据库响应表中。(其中涉及姓名的信息要有一条和自己的姓名相关的信息)
        \item 自己各构造10条学生交费信息表和学生学籍信息表的数据,存在名字为“学生交费信.xls”和“学生学籍.xls”的Excel文件中,然后导入到数据库响应表中。(其中涉及姓名的信息要有一条和自己的姓名相关的信息)
        \item 在视图界面中,向课程信息表的添加10条数据,自己各构造。
        \item 在视图界面中,导出数据表信息,包含Excel格式和TXT文件格式。
    \end{enumerate}

    \subsection{命令方式建立数据表}

    向实验三教材第三章的例子用到的三张表添加数据:

    \begin{enumerate}
        \item 用Insert 命令向表(Course )中插入书上教材第三章例子中的数据。
        \item 将书上教材第三章例子中的表(Student)中存入“Student.xls”文件中,然后用命令导入数据库相应的表。
        \item 将书上教材第三章例子中的表(SC)中存入“SC.txt”文件中,然后用命令导入数据库相应的表。
        \item 以命令方式,导出数据表信息,包含Excel格式和TXT文件格式。
    \end{enumerate}

    \section{实验总结}
\end{document}